\chapter{Various}

\section{Intervals}
	\kactlimport{IntervalContainer.h}
	\kactlimport{IntervalCover.h}
	\kactlimport{ConstantIntervals.h}

\section{Misc. algorithms}
	%\kactlimport{TernarySearch.h}
	\kactlimport{Karatsuba.h}
	\kactlimport{Josephus.h}

\section{Dynamic programming}
	\kactlimport{DivideAndConquerDP.h}
	\kactlimport{KnuthDP.h}
  \kactlimport{wqsBinaryDP.h}

% \section{Java}
    % \kactlimport[-l raw]{java.java}

\section{Debugging tricks}
	\begin{itemize}
		\item \texttt{signal(SIGSEGV, [](int) \{ \_Exit(0); \});} converts segfaults into Wrong Answers.
			Similarly one can catch SIGABRT (assertion failures) and SIGFPE (zero divisions).
			\texttt{\_GLIBCXX\_DEBUG} violations generate SIGABRT (or SIGSEGV on gcc 5.4.0 apparently).
		\item \texttt{feenableexcept(29);} kills the program on NaNs (\texttt 1), 0-divs (\texttt 4), infinities (\texttt 8) and denormals (\texttt{16}).
	\end{itemize}

\section{Optimization tricks}
	\subsection{Bit hacks}
		\begin{itemize}
			\item \texttt{x \& -x} is the least bit in \texttt{x}.
            \item \texttt{x \&\& !(x \& (x - 1))} true, if \texttt{x} is power of 2.
            \item \texttt{gray\_code[x] = x \^{} (x >> 1)}
            \item \texttt{checkerboard[y][x] = (x \& 1) \^{} (y \& 1)}
            \item \texttt{ffs(int x), ffs(ll x)} number of the least significant bit, \texttt{ffs(1 << i) = i+1}
            \item \texttt{\_\_builtin\_clz(uint x), ...\_clzll(ull)} number of leading zeros, for x > 0
            \item \texttt{\_\_builtin\_ctz(uint x), ...\_ctzll(ull)} number of trailing zeros, for x > 0
            \item \texttt{\_\_builtin\_popcount(uint x), ...\_popcountll(ull)} number of 1 bits
            \item \texttt{\#define ld\_ll(X) (63-\_\_builtin\_clzll(ll(X)))} floor(log2(X))
			\item \texttt{for (int x = m; x; ) \{ --x \&= m; ... \}} loops over all subset masks of \texttt{m} (except \texttt{m} itself).
			\item \texttt{c = x\&-x, r = x+c; (((r\^{}x) >> 2)/c) | r} is the next number after \texttt{x} with the same number of bits set.
			\item \texttt{ FOR(b,0,K) FOR(i,0,(1 << K)) if (i \& 1 << b) D[i] += D[i\^{}(1 << b)]; } computes all sums of subsets.
		\end{itemize}
        \subsection{Pragmas}
		\begin{itemize}
        \item \lstinline{#pragma GCC optimize ("Ofast")} will make GCC auto-vectorize for loops and optimizes floating points better (assumes associativity and turns off denormals).
        \item \lstinline{#pragma GCC target ("avx,avx2")} can double performance of vectorized code, but causes crashes on old machines.
        \item \lstinline{#pragma GCC optimize ("trapv")} kills the program on integer overflows (but is really slow).
		\end{itemize}

        \subsection{Random Problem Solutions}
        \subsubsection{Strings}
        \begin{itemize}
        \item Distinct substrings in a word: Sum len(q) - len(suffixlink(q))
        \item Sum of Lengths of distincst substrings: We count how many times a letter appears in distinct substrings, as all paths in automata ara different substrings we have 1 + number of paths, we can solve this by easy dp.
        \end{itemize}
        \subsubsection{$n$-Queens Problem $n\geq 4$ or $n=1$}
        \begin{enumerate}
        \item Divide $n$ by 12. Remember the remainder ($n$ is 8 for the eight queens puzzle).
        \item Write a list of the even numbers from 2 to n in order. If the remainder is 3 or 9, move 2 to the end of the list.
        \item Append the odd numbers from 1 to $n$ in order, but, if the remainder is 8, switch pairs (i.e. 3, 1, 7, 5, 11, 9, \ldots).
        \item If the remainder is 2, switch the places of 1 and 3, then move 5 to the end of the list.
        \item If the remainder is 3 or 9, move 1 and 3 to the end of the list.
        \item Place the first-column queen in the row with the first number in the list,  place the second-column queen in the row with the second number in the list, etc.
        \end{enumerate}
        For $n = 8$ this results in the solution shown above. A few more examples follow.
        \begin{itemize}
        \item 14 queens (remainder 2): 2, 4, 6, 8, 10, 12, 14, 3, 1, 7, 9, 11, 13, 5.
        \item 15 queens (remainder 3): 4, 6, 8, 10, 12, 14, 2, 5, 7, 9, 11, 13, 15, 1, 3.
        \item 20 queens (remainder 8): 2, 4, 6, 8, 10, 12, 14, 16, 18, 20, 3, 1, 7, 5, 11, 9, 15, 13, 19, 17.
        \end{itemize}

\subsubsection{Solutions}
\begin{tabular}{|c|c|}
    \hline
    Usar totes les dades & random/clock \\
    \hline
    DP(+bitmask) & binaria \\
    \hline
    Sparse backtracking & greedy \\
    \hline
    Interval Tree& FFT \\
    \hline
    Bullir l'olla & MaxFlow \\
    \hline
    Suffix Tree& MinCut \\
    \hline
    Dibuixar 2D al pla & Precalcul \\
    \hline
    aux vectors in DP& 2-SAT \\
    \hline
    pattern & hashing \\
    \hline
    think it's easy & sweepline\\
    \hline
    look spec. cases & \\
    \hline
    & \\
    \hline
    & \\
    \hline
\end{tabular}
