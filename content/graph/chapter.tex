\chapter{Graph}
\kactlimport{ArticulationPointAndBridges.h}
\kactlimport{EulerianCycle.h} %tested
\kactlimport{SCC.h} %tested
\kactlimport{2sat.h} 
\kactlimport{BiconnectedComponents.h} 
\kactlimport{MinimalArborescence.h}
\kactlimport{BinaryJumps.h}
\kactlimport{TreePower.h}
\kactlimport{LinkCutTree.h}
\kactlimport{DominatorTree.h}
\kactlimport{FastMaxFlow.h}
\kactlimport{PushRelabel.h}
\kactlimport{MaxflowMinCap.h}
\kactlimport{MaxflowMinCost.h}
\kactlimport{StableMarriage.h}
\kactlimport{DFSMatching.h}
\kactlimport{GeneralMatching.h}
% \kactlimport{MaxWeightBipartiteMatching.h}
\kactlimport{MinCostBipartite.h} %Tested mincost and maxcost
\kactlimport{globalmincut.h}
\section{Karp's Minimum Mean Cycle}
1. Initialize: $d_0(r) = 0$ and $\forall v\in V - \{r\}$,$d_0(v) = \infty$.\\
2. For $i = 1$ to $n$:\\
--- For every $v\in V$ calculate:\\
----- $d_i(v) = \min_{u:(u,v)\in E} = \{d_{i - 1}(u) + c((u,v))\}$\\
Optim = $\min_{v\in V}\max_{0\leq k\leq n - 1}\left\{\frac{d_n(v) - d_k(v)}{n - k}\right\}$
\section{Planar Graphs}
\textbf{Formula Euler}: v-e+f = 2\\
\textbf{Theorem 1}: If $v>= 3$ then $e<=3v-6$.\\
\textbf{Theorem 2}: If $v>3$ and there are no cycles of length 3, then $e<= 2v-4$.\\

\section{Bipartite Graphs}
let G = (V,E) be a Bipartite graph, we will call each part L and R respectively.\\
\textbf{Maximum Matching} is equal to \textbf{Minimum Vertex Cover}, to get the nodes in the MVC we choose the unmatched vertices of L
and run a BFS/DFS in the graph with edges in the matching directed from R to L, and edges not in the matching directed from L to R. Let's all this Z,
now MVC = (L\textbackslash Z) $\cup$ (R $\cap$ Z). \\
The \textbf{Minimum Edge Cover} is obtained by doing a Maximum Matching, Then run a BFS/DFS from unmatched vertices of L,
the MEC is the unreachable vertices of A and reachable vertices of B.\\
The \textbf{Maximum Independent Set} is the complementary of the \textbf{Minimum Vertex Cover}
\subsection{DAG}
The \textbf{Minimum Path Cover} is given by the edges in the MM of the bipartite graph after doubling the vertices.
\section{ASSP : Johnson Algorithm}
We have a sparse graph with, possibly, negative edges. We want to compute the all-pairs shortest path. Floyd Warshall may be too slow.\\
Algorithm: \\
1) A new node q is added to the graph, connected by zero-weight edges to each of the other nodes.\\
2) Bellman–Ford algorithm is used, starting from the new vertex q, to find for each vertex v the minimum weight h(v) of a path from q to v. \\
If this step detects a negative cycle, the algorithm is terminated as the shortest-path is undefined.\\
3) edges of the original graph are reweighted using the values computed by the Bellman–Ford algorithm: \\
an edge from u to v, having length w(u,v), is given by the new length w(u,v) + h(u) - h(v).\\
4) q is removed, and Dijkstra's algorithm is used to find the shortest paths from each node s to every other vertex in the reweighted graph.\\
5) Dijkstra from every node. The shortest path in this new graph is the same as in the initial graph although it's weight changes.\\
The idea is to do dijkstra by storing "weight in new graph" and "weight in old graph" sorting by the first one.\\
